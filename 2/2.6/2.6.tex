\documentclass{beamer}

\mode<presentation> {

%\usetheme{default}
%\usetheme{AnnArbor}
%\usetheme{Antibes}
%\usetheme{Bergen}
%\usetheme{Berkeley}
%\usetheme{Berlin}
%\usetheme{Boadilla}
%\usetheme{CambridgeUS}
%\usetheme{Copenhagen}
%\usetheme{Darmstadt}
%\usetheme{Dresden}
%\usetheme{Frankfurt}
%\usetheme{Goettingen}
%\usetheme{Hannover}
%\usetheme{Ilmenau}
%\usetheme{JuanLesPins}
%\usetheme{Luebeck}
\usetheme{Madrid}
%\usetheme{Malmoe}
%\usetheme{Marburg}
%\usetheme{Montpellier}
%\usetheme{PaloAlto}
%\usetheme{Pittsburgh}
%\usetheme{Rochester}
%\usetheme{Singapore}
%\usetheme{Szeged}
%\usetheme{Warsaw}


%\usecolortheme{albatross}
%\usecolortheme{beaver}
%\usecolortheme{beetle}
%\usecolortheme{crane}
%\usecolortheme{dolphin}
%\usecolortheme{dove}
%\usecolortheme{fly}
%\usecolortheme{lily}
%\usecolortheme{orchid}
%\usecolortheme{rose}
%\usecolortheme{seagull}
%\usecolortheme{seahorse}
%\usecolortheme{whale}
%\usecolortheme{wolverine}

%\setbeamertemplate{footline} % To remove the footer line in all slides uncomment this line
%\setbeamertemplate{footline}[page number] % To replace the footer line in all slides with a simple slide count uncomment this line

%\setbeamertemplate{navigation symbols}{} % To remove the navigation symbols from the bottom of all slides uncomment this line
}

\usepackage{graphicx} % Allows including images
\usepackage{booktabs} % Allows the use of \toprule, \midrule and \bottomrule in tables
\usepackage{amsfonts}
\usepackage{mathrsfs}
\usepackage{amsmath,amssymb,graphicx}

%----------------------------------------------------------------------------------------
%	TITLE PAGE
%----------------------------------------------------------------------------------------

\title["2.6"]{2.6: The Wold Decomposition}

\author{Taylor} 
\institute[UVA] 
{
University of Virginia \\
\medskip
\textit{} 
}
\date{} 

\begin{document}
%----------------------------------------------------------------------------------------

\begin{frame}
\titlepage 
\end{frame}
%----------------------------------------------------------------------------------------

\begin{frame}
\frametitle{Motivation}

The Wold Decomposition tells us that any stationary time series can be broken up into a deterministic part and a linear process (like an ARMA process).

In a sense, this justifies what we were doing in the first chapter: ARMA models are good for the non-deterministic part.

\end{frame}

%----------------------------------------------------------------------------------------

\begin{frame}
\frametitle{Definition}

A {\bf deterministic} process is one that has no prediction error. In other words, $X_t = \tilde{P}_{t-1}X_t$.
\newline

Example 1: $X_t = A \cos(\omega t) + B \sin(\omega t)$, where $\omega \in (0,\pi)$, and $A$ and $B$ are uncorrelated but have mean $0$ and variance $\sigma^2$. 
\newline

Example 2: $X_t = A + b t$

\end{frame}

%----------------------------------------------------------------------------------------

\begin{frame}
\frametitle{Definition}

\begin{enumerate}
\item $X_t = A \cos(\omega t) + B \sin(\omega t)$ 
\item $X_t = A + b t$
\end{enumerate}
It's easier to see if you write the difference equations. There's no noise!
\newline

Example 1: 
\begin{align*}
X_t &= [2 \cos(\omega)] X_{t-1} - X_{t-2} & \tag{trig identities}\\
&= \tilde{P}_{t-1} X_t.
\end{align*}

Example 2: 
\begin{align*}
X_t &= X_t - X_{t-1} +X_{t-1} \\
&= b + X_{t-1}
&= \tilde{P}_{t-1} X_t.
\end{align*}

\end{frame}


% %----------------------------------------------------------------------------------------

\begin{frame}
\frametitle{Notation}

Two potentially confusing things in this chapter:
\newline

Here $\tilde{P}$ denotes the prediction operator for a time series with an infinite past. We skipped this section. Don't worry about the slight difference between this and $P_{t-1}$. 
\newline

Also, deterministic does not mean non-random! This can be seen from the last two examples.
\end{frame}

%----------------------------------------------------------------------------------------

\begin{frame}
\frametitle{Theorem}

\begin{block}{The Wold Decomposition}
If $\{X_t\}$ is a nondeterministic stationary time series, then it can be written as
\[
X_t = \sum_{j=0}^{\infty}\psi_j Z_{t-j} + V_t
\]
where
\begin{itemize}
\item $\psi_0 = 1$ and $\sum_{j=0}^{\infty}\psi_j^2 < \infty$
\item $\{Z_t\} \sim \text{WN}(0,\sigma^2)$
\item $\text{Cov}(Z_s,V_t) = 0$ for all $s$ and $t$
\item $Z_t = \tilde{P}_t Z_t$ for all $t$ (invertibility)
\item $V_t = \tilde{P}_s V_t$ for all $s$ and $t$ 
\item $\{V_t\}$ is deterministic.
\end{itemize}
\end{block}

\end{frame}

%----------------------------------------------------------------------------------------

\begin{frame}
\frametitle{Example 1}

Is the following stationary? What are the deterministic and non-deterministic pieces?
\begin{align*}
Y_t &= A\cos(\omega t) + B \sin(\omega t) + X_t\\
\phi(B) X_t &= \theta(B) Z_t
\end{align*}


\end{frame}

%----------------------------------------------------------------------------------------

\begin{frame}
\frametitle{Example 2}

Is the following stationary? What are the deterministic and non-deterministic pieces?
\begin{align*}
Y_t &= \sum_i \left[ A_i \cos(\omega_i t) + B_i \sin(\omega_i t) \right]+ X_t\\
\phi(B) X_t &= \theta(B) Z_t
\end{align*}


\end{frame}



%----------------------------------------------------------------------------------------

\begin{frame}
\frametitle{Example 3}


Is the following stationary? What are the deterministic and non-deterministic pieces?

\[
Y_t = \beta_0 + \beta_1 x_t + \epsilon_t
\]
\[
\phi(B) \epsilon_t = \theta(B) Z_t
\]
\end{frame}


\end{document} 