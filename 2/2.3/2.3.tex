\documentclass{beamer}

\mode<presentation> {

%\usetheme{default}
%\usetheme{AnnArbor}
%\usetheme{Antibes}
%\usetheme{Bergen}
%\usetheme{Berkeley}
%\usetheme{Berlin}
%\usetheme{Boadilla}
%\usetheme{CambridgeUS}
%\usetheme{Copenhagen}
%\usetheme{Darmstadt}
%\usetheme{Dresden}
%\usetheme{Frankfurt}
%\usetheme{Goettingen}
%\usetheme{Hannover}
%\usetheme{Ilmenau}
%\usetheme{JuanLesPins}
%\usetheme{Luebeck}
\usetheme{Madrid}
%\usetheme{Malmoe}
%\usetheme{Marburg}
%\usetheme{Montpellier}
%\usetheme{PaloAlto}
%\usetheme{Pittsburgh}
%\usetheme{Rochester}
%\usetheme{Singapore}
%\usetheme{Szeged}
%\usetheme{Warsaw}


%\usecolortheme{albatross}
%\usecolortheme{beaver}
%\usecolortheme{beetle}
%\usecolortheme{crane}
%\usecolortheme{dolphin}
%\usecolortheme{dove}
%\usecolortheme{fly}
%\usecolortheme{lily}
%\usecolortheme{orchid}
%\usecolortheme{rose}
%\usecolortheme{seagull}
%\usecolortheme{seahorse}
%\usecolortheme{whale}
%\usecolortheme{wolverine}

%\setbeamertemplate{footline} % To remove the footer line in all slides uncomment this line
%\setbeamertemplate{footline}[page number] % To replace the footer line in all slides with a simple slide count uncomment this line

%\setbeamertemplate{navigation symbols}{} % To remove the navigation symbols from the bottom of all slides uncomment this line
}

\usepackage{graphicx} % Allows including images
\usepackage{booktabs} % Allows the use of \toprule, \midrule and \bottomrule in tables
\usepackage{amsfonts}
\usepackage{mathrsfs}
\usepackage{amsmath,amssymb,graphicx}

%----------------------------------------------------------------------------------------
%	TITLE PAGE
%----------------------------------------------------------------------------------------

\title["2.3"]{2.3: Introduction to ARMA Processes}

\author{Taylor} 
\institute[UVA] 
{
University of Virginia \\
\medskip
\textit{} 
}
\date{} 

\begin{document}
%----------------------------------------------------------------------------------------

\begin{frame}
\titlepage 
\end{frame}
%----------------------------------------------------------------------------------------

\begin{frame}
\frametitle{Motivation}

We introduce Autoregressive Moving Average Processes (ARMA) by doing some examples. Through these examples we start talking about some of the ``weird" things that can happen with these, which supports our hypothesis that time series modelling is more difficult than vanilla regression.

\end{frame}

%----------------------------------------------------------------------------------------

\begin{frame}
\frametitle{Definition}

\begin{block}{ARMA(1,1)}
The time series $\{X_t\}$ is an {\bf ARMA(1,1)} process if it is stationary and it satisfies
\[
X_t - \phi X_{t-1} = Z_t + \theta Z_{t-1},
\]
where $\{Z_t\} \sim \text{WN}(0,\sigma^2)$ and $\phi + \theta \neq 0$
\end{block}
this can be written as $\phi(B)X_t = \theta(B)(Z_t)$
\end{frame}

%----------------------------------------------------------------------------------------


\begin{frame}
\frametitle{Restrictions on Parameters}
Why  $\phi + \theta \neq 0$? Say $|\phi|<1$, then
\begin{align*}
X_t &= \phi^{-1}(B)\theta(B)Z_t \\
&= (1 + \phi B + \phi^2 B^2 + \phi^3B^3  + \cdots)(1 + \theta B)Z_t\\
&= [1 + (\theta + \phi)B + (\phi\theta + \phi^2 )B^2 + (\phi^2 \theta+\phi^3)B^3\cdots]Z_t \\
&= Z_t + \left[ (\theta + \phi)\sum_{j=1}^{\infty} \phi^{j-1} B^j \right] Z_t
\end{align*}
Show that if $|\phi|>1$ then
\[
X_t = -\theta\phi^{-1}Z_t - (\theta+\phi) \sum_{j=1}^{\infty} \phi^{-(j+1)}Z_{t+j}
\]
\end{frame}

%----------------------------------------------------------------------------------------


\begin{frame}
\frametitle{Summary and New Term}

Note:
\begin{enumerate}
\item Stationary solution exists iff $\phi \neq \pm 1$
\item Solution is {\bf causal} iff $|\phi| < 1$
\item Solution is {\bf noncausal} if $|\phi|>1$
\end{enumerate}

New definition: invertibility.
\begin{enumerate}
\item $X_t$ is {\bf causal } if it can be written in terms of $Z_s, s \le t$
\item $X_t$ is {\bf invertible} if $Z_t$ can be written in terms of $X_s, s \le t$
\end{enumerate}


\end{frame}

%----------------------------------------------------------------------------------------


\begin{frame}
\frametitle{Invertibility Demo}

ARMA(1,1) is invertible iff $|\theta|<1$.
\begin{align*}
Z_t &= \theta^{-1}(B)\phi(B)X_t \\
&= (1 -\theta B + \theta^2B^2 - \theta^{3}B^3 + \cdots)(1 - \phi B) \\
&= [1 - (\phi +\theta)B + (\theta\phi + \theta^2)B^2 + \cdots ] X_t \\
&\vdots\\
&= X_t - (\theta + \phi) \sum_{j=1}^{\infty} (-\theta)^{j-1}X_{t-j}
\end{align*}
Show if $|\theta|>1$ then it isn't invertible and  
\[
Z_t = -\phi \theta^{-1} X_t + (\theta + \phi) \sum_{j=1}^{\infty} (-\theta)^{-(j-1)}X_{t+j}.
\]
\end{frame}

%----------------------------------------------------------------------------------------

%----------------------------------------------------------------------------------------


\begin{frame}
\frametitle{Summary}

\begin{enumerate}
\item $\phi + \theta \neq 0$
\item Stationary solution exists iff $|\phi| \neq 1$.
\item Solution is {\bf causal} iff $|\phi| < 1$
\item Solution is {\bf invertible} if $|\theta| < 1$
\end{enumerate}

The errors and data are linear combinations of each other. Causal and invertible models are preferable because the alternatives don't really ``make sense," especially in the context of financial data.

\end{frame}




\end{document} 