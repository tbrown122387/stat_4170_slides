\documentclass{beamer}

\mode<presentation> {

%\usetheme{default}
%\usetheme{AnnArbor}
%\usetheme{Antibes}
%\usetheme{Bergen}
%\usetheme{Berkeley}
%\usetheme{Berlin}
%\usetheme{Boadilla}
%\usetheme{CambridgeUS}
%\usetheme{Copenhagen}
%\usetheme{Darmstadt}
%\usetheme{Dresden}
%\usetheme{Frankfurt}
%\usetheme{Goettingen}
%\usetheme{Hannover}
%\usetheme{Ilmenau}
%\usetheme{JuanLesPins}
%\usetheme{Luebeck}
\usetheme{Madrid}
%\usetheme{Malmoe}
%\usetheme{Marburg}
%\usetheme{Montpellier}
%\usetheme{PaloAlto}
%\usetheme{Pittsburgh}
%\usetheme{Rochester}
%\usetheme{Singapore}
%\usetheme{Szeged}
%\usetheme{Warsaw}


%\usecolortheme{albatross}
%\usecolortheme{beaver}
%\usecolortheme{beetle}
%\usecolortheme{crane}
%\usecolortheme{dolphin}
%\usecolortheme{dove}
%\usecolortheme{fly}
%\usecolortheme{lily}
%\usecolortheme{orchid}
%\usecolortheme{rose}
%\usecolortheme{seagull}
%\usecolortheme{seahorse}
%\usecolortheme{whale}
%\usecolortheme{wolverine}

%\setbeamertemplate{footline} % To remove the footer line in all slides uncomment this line
%\setbeamertemplate{footline}[page number] % To replace the footer line in all slides with a simple slide count uncomment this line

%\setbeamertemplate{navigation symbols}{} % To remove the navigation symbols from the bottom of all slides uncomment this line
}

\usepackage{graphicx} % Allows including images
\usepackage{booktabs} % Allows the use of \toprule, \midrule and \bottomrule in tables
\usepackage{amsfonts}
\usepackage{mathrsfs}
\usepackage{amsmath,amssymb,graphicx, bm}
\usepackage{dirtytalk} % quote thing

%----------------------------------------------------------------------------------------
%	TITLE PAGE
%----------------------------------------------------------------------------------------

\title["8.7"]{8.7: Cointegration}

\author{Taylor} 
\institute[UVA] 
{
University of Virginia \\
\medskip
\textit{} 
}
\date{} 

\begin{document}
%----------------------------------------------------------------------------------------

\begin{frame}
\titlepage 
\end{frame}
%----------------------------------------------------------------------------------------

\begin{frame}
\frametitle{Motivation}

\begin{enumerate}
\item What happens when the roots of $\det[\bm{\Phi}(z)]$ are not all outside the unit circle?
\item What happens if several series share a stochastic trend, or in other words, if there is a long-run economic equilibrium between several non-stationary variables?
\end{enumerate}

Supplementary materials: chapters 6.3 of ``New Introduction to Multiple Time Series Analysis" by Helmut L\"{u}tkepohl.

\end{frame}

%----------------------------------------------------------------------------------------

\begin{frame}
\frametitle{Example 8.7.1}

If
\begin{align*}
\left[\begin{array}{c}
X_t \\
Y_t
\end{array}\right]
&=
\left[\begin{array}{c}
\sum_{i=1}^t Z_t \\
\sum_{i=1}^t Z_t + W_t
\end{array}\right] \\
&=
\left[ \begin{array}{cc}
1 & 0 \\
1 & 1
\end{array}\right]
\left[\begin{array}{c}
Z_t \\
W_t
\end{array}\right] 
+
\left[ \begin{array}{cc}
1 & 0 \\
1 & 0
\end{array}\right]
\left[\begin{array}{c}
Z_{t-1} \\
W_{t-1}
\end{array}\right]+ \cdots
\end{align*}
then
\begin{align*}
\left[\begin{array}{c}
X_t \\
Y_t
\end{array}\right]
&=
\left[\begin{array}{c}
X_{t-1} \\
Y_{t-1}
\end{array}\right]
+
\left[\begin{array}{cc}
1 & 0 \\
1 & 1
\end{array}\right]
\left[\begin{array}{c}
Z_{t} \\
W_{t}
\end{array}\right]
+
\left[\begin{array}{cc}
0 & 0 \\
0 & -1
\end{array}\right]
\left[\begin{array}{c}
Z_{t-1} \\
W_{t-1}
\end{array}\right]
\end{align*}
\pause

$\det[\bm{\Phi}(z)] = \det[I - Iz] = (1-z)(1-z)$. The roots are $z_1 = z_2 = 1$.



\end{frame}

%----------------------------------------------------------------------------------------

\begin{frame}
\frametitle{Example 8.7.1}

So let's try differencing all the series. 
\begin{align*}
\left[\begin{array}{c}
X_t \\
Y_t
\end{array}\right]
&=
\left[\begin{array}{c}
X_{t-1} \\
Y_{t-1}
\end{array}\right]
+
\left[\begin{array}{cc}
1 & 0 \\
1 & 1
\end{array}\right]
\left[\begin{array}{c}
Z_{t} \\
W_{t}
\end{array}\right]
+
\left[\begin{array}{cc}
0 & 0 \\
0 & -1
\end{array}\right]
\left[\begin{array}{c}
Z_{t-1} \\
W_{t-1}
\end{array}\right]
\end{align*}
becomes
$$
\triangledown 
\left[\begin{array}{c}
X_t \\
Y_t
\end{array}\right]
=
\left[\begin{array}{cc}
1 & 0 \\
1 & 1
\end{array}\right]
\left[\begin{array}{c}
Z_{t} \\
W_{t}
\end{array}\right]
+
\left[\begin{array}{cc}
0 & 0 \\
0 & -1
\end{array}\right]
\left[\begin{array}{c}
Z_{t-1} \\
W_{t-1}
\end{array}\right]
$$
\pause

$\det[\bm{\Theta}(z)] = 1+z$. This process is not invertible. Note that you need to re-define the noise as $(Z_t, Z_{t} + W_t)'$ so that the leading coefficient matrix is an identity matrix.
\end{frame}

%----------------------------------------------------------------------------------------

\begin{frame}
\frametitle{Example 8.7.1}

\begin{align*}
\left[\begin{array}{c}
X_t \\
Y_t
\end{array}\right]
&=
\left[\begin{array}{c}
\sum_{i=1}^t Z_t \\
\sum_{i=1}^t Z_t + W_t
\end{array}\right] \\
&=
\left[ \begin{array}{cc}
1 & 0 \\
1 & 1
\end{array}\right]
\left[\begin{array}{c}
Z_t \\
W_t
\end{array}\right] 
+
\left[ \begin{array}{cc}
1 & 0 \\
1 & 0
\end{array}\right]
\left[\begin{array}{c}
Z_{t-1} \\
W_{t-1}
\end{array}\right]+ \cdots
\end{align*}
Notice that $X_t = Y_t + W_t$, so $X_t \approx Y_t$. They are each separately nonstationary, but they are ``tied together." If you didn't difference them, but instead looked at the spread between them, this would be an AR(0) model.

\end{frame}

%----------------------------------------------------------------------------------------

\begin{frame}
\frametitle{Definitions}

We call a $K$-dimensional process $Y_t$ {\bf integrated of order $d$}, written as $Y_t \sim I(d)$, if $\triangledown^d Y_t$ is causal and stationary and $\triangledown^{d-1} Y_t$ is not. Usually we will be interested in the case of $d = 1$.
\newline

An $I(d)$ process $Y_t$ is called {\bf cointegrated}  if there exists atleast one (there may be more) linear combination $\beta'Y_t$, $\beta \neq 0$ which is integrated of order less than $d$.
\newline

In the last example $d=1$ and $\beta' = (1,-1)$. Often there will be more than $1$ linear combination (e.g. $\bm{\beta}$). And these vectors are not unique.


\end{frame}


%----------------------------------------------------------------------------------------

\begin{frame}
\frametitle{Definitions}

\begin{block}{VECM}
Let $\mathbf{Y}_t$ be a $K$-dimensional, vector-valued time series. A {\bf vector error correction model} (VECM) can be written as
$$
\triangledown Y_t = \bm{\alpha} \bm{\beta} ' Y_{t-1} + \Gamma_1 \triangledown Y_{t-1} + U_t,
$$
where $\bm{\alpha}$, $\bm{\beta}$ are $K \times r$ matrices.
\end{block}

This is like a VAR(1) model, but note that the differences depend additionally on linear combinations of the levels: $ \bm{\beta} ' Y_{t-1}$. 
\newline

Alternatively you could think of this as a nonstationary VAR(2) model (see next slide).

\end{frame}

%----------------------------------------------------------------------------------------

\begin{frame}
\frametitle{VAR vs VECM}

Start with a non-stationary VAR(2)
$$
Y_t = A_1 Y_{t-1} + A_2 Y_{t-2} + U_t
$$
difference once
\begin{align*}
Y_t - Y_{t-1} &= -IY_{t-1} + A_1 Y_{t-1} + A_2 Y_{t-2} + U_t \\
 &= -IY_{t-1} + A_1 Y_{t-1} + A_2 Y_{t-1} - A_2 Y_{t-1} + A_2 Y_{t-2} + U_t\\
 &= -(I - A_1 - A_2)Y_{t-1} - A_2 \triangledown Y_{t-1} + W_t \\
 &= \bm{\alpha} \bm{\beta} ' Y_{t-1} + \Gamma_1 \triangledown Y_{t-1} + U_t
\end{align*}

Note that -$(I - A_1 - A_2) = \bm{\alpha} \bm{\beta} '$ is singular because $\det[ \bm{\Phi}(z)]$ has a unit root. 

\end{frame}


%----------------------------------------------------------------------------------------

\begin{frame}
\frametitle{A Second Definition}

The $K$-dimensional VAR(p)
$$
Y_t = A_1 Y_{t-1} + A_2 Y_{t-2} + \cdots + A_pY_{t-p} + U_t
$$
is {\bf cointegrated of rank $r$} if 
$$
\bm{\Pi} \overset{\text{def}}{=} -(I - A_1 - \cdots - A_p)
$$
has rank $0 < r < K$, and thus can be written as $\bm{\alpha} \bm{\beta}'$ with each of these matrices being of dimension $K \times r$. The matrix $\bm{\beta}$ is called the {\bf cointegration matrix} or {\bf matrix of cointegrating vectors} and $\bm{\alpha}$ is called the {\bf loadings matrix}. 

\end{frame}



%----------------------------------------------------------------------------------------

\begin{frame}
\frametitle{A Little Algebra}

Assume the $K$-dimensional VAR(p)
$$
Y_t = A_1 Y_{t-1} + A_2 Y_{t-2} + \cdots + A_pY_{t-p} + U_t
$$
is cointegrated of rank $r$. Then 
\begin{align*}
\triangledown Y_t &= - IY_{t-1} + A_1 Y_{t-1} + A_2 Y_{t-2} + \cdots + A_pY_{t-p} + U_t\\
&= - IY_{t-1} + A_1 Y_{t-1} + A_2 Y_{t-2} + \cdots + A_pY_{t-p} + U_t \\
& \hspace{10mm} \pm A_2 Y_{t-1} \pm \cdots \pm A_p Y_{t-1} \\
&= \bm{\Pi} Y_{t-1} + (-A_2 - A_3 - \cdots - A_p) Y_{t-1} \\
& \hspace{10mm} A_2 Y_{t-2} + \cdots + A_p Y_{t-p} + U_t\\
&= \bm{\Pi} Y_{t-1} + \bm{\Gamma}_1\triangledown Y_{t-1} + \cdots + \bm{\Gamma}_{p-1} \triangledown Y_{t-p+1} 
\end{align*}
where $A_1 = \bm{\Pi} + I + \bm{\Gamma}_1$, $A_p = -\bm{\Gamma}_{p-1}$ and $A_i = \bm{\Gamma}_i - \bm{\Gamma}_{i-1}$ for $i=2,\ldots, p-1$.

\end{frame}

%----------------------------------------------------------------------------------------

\begin{frame}
\frametitle{Interpretation of $\bm{\beta} ' Y_{t-1}$}

In
$$
\triangledown Y_t = \bm{\alpha} \bm{\beta} ' Y_{t-1} + \Gamma_1 \triangledown Y_{t-1} + U_t,
$$

$\bm{\alpha} \bm{\beta} ' Y_{t-1} $  must be causal and stationary. Therefore
$$ 
[\bm{\alpha}'\bm{\alpha}] ^{-1}\bm{\alpha}'\bm{\alpha} \bm{\beta} ' Y_{t-1} = \bm{\beta} ' Y_{t-1}
$$ 
must be causal and stationary. 
\newline

Think of this as a lower-dimensional series of ``spreads" between financial time series. They are not differenced, but they are linearly-combined in $1$ or more ways.

\end{frame}

%----------------------------------------------------------------------------------------

\begin{frame}
\frametitle{Nota Bene: Non-Uniqueness of Cointegrating Relationships}

The matrix $\bm{\Pi} = \bm{\alpha}\bm{\beta}'$ is not unique. Take any invertible matrix $Q$. Clearly 
$$
\bm{\alpha}\bm{\beta}' = \bm{\alpha}Q Q^{-1}\bm{\beta}',
$$
so we can always set $\bm{\alpha}^* = \bm{\alpha}Q$ and $\bm{\beta}^* =\bm{\beta} Q^{-1}$


\end{frame}



\end{document} 