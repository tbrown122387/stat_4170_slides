\documentclass{beamer}

\mode<presentation> {

%\usetheme{default}
%\usetheme{AnnArbor}
%\usetheme{Antibes}
%\usetheme{Bergen}
%\usetheme{Berkeley}
%\usetheme{Berlin}
%\usetheme{Boadilla}
%\usetheme{CambridgeUS}
%\usetheme{Copenhagen}
%\usetheme{Darmstadt}
%\usetheme{Dresden}
%\usetheme{Frankfurt}
%\usetheme{Goettingen}
%\usetheme{Hannover}
%\usetheme{Ilmenau}
%\usetheme{JuanLesPins}
%\usetheme{Luebeck}
\usetheme{Madrid}
%\usetheme{Malmoe}
%\usetheme{Marburg}
%\usetheme{Montpellier}
%\usetheme{PaloAlto}
%\usetheme{Pittsburgh}
%\usetheme{Rochester}
%\usetheme{Singapore}
%\usetheme{Szeged}
%\usetheme{Warsaw}


%\usecolortheme{albatross}
%\usecolortheme{beaver}
%\usecolortheme{beetle}
%\usecolortheme{crane}
%\usecolortheme{dolphin}
%\usecolortheme{dove}
%\usecolortheme{fly}
%\usecolortheme{lily}
%\usecolortheme{orchid}
%\usecolortheme{rose}
%\usecolortheme{seagull}
%\usecolortheme{seahorse}
%\usecolortheme{whale}
%\usecolortheme{wolverine}

%\setbeamertemplate{footline} % To remove the footer line in all slides uncomment this line
%\setbeamertemplate{footline}[page number] % To replace the footer line in all slides with a simple slide count uncomment this line

%\setbeamertemplate{navigation symbols}{} % To remove the navigation symbols from the bottom of all slides uncomment this line
}

\usepackage{graphicx} % Allows including images
\usepackage{booktabs} % Allows the use of \toprule, \midrule and \bottomrule in tables
\usepackage{amsfonts}
\usepackage{mathrsfs}
\usepackage{amsmath,amssymb,graphicx}

%----------------------------------------------------------------------------------------
%	TITLE PAGE
%----------------------------------------------------------------------------------------

\title["2.2"]{2.2: Axioms, Interpretations and Properties of Probability}

\author{Taylor} 
\institute[UVA] 
{
University of Virginia \\
\medskip
\textit{} 
}
\date{} 

\begin{document}
%----------------------------------------------------------------------------------------

\begin{frame}
\titlepage 
\end{frame}
%----------------------------------------------------------------------------------------

\begin{frame}
\frametitle{Motivation}

The sets are basically questions we can ask about something random...but how do we talk about the chances that those things will occur? A: we need a function that takes events as input, and as output gives us a number between $0$ and $1$. This is our probability function $P( \cdot ) $. 
\newline


\end{frame}

%----------------------------------------------------------------------------------------

\begin{frame}
\frametitle{Properties of $P$}

Here are the rules any probability function has to satisfy;

\begin{enumerate}
\item For any event $A$, $P(A) \ge 0$
\item $P(\mathcal{S}) = 1$
\item let $A_1, A_2, \ldots$ be a countably infinite collection of disjoint sets; then $P(A_1 \cup A_2 \cup \cdots ) = \sum_{i=1}^{\infty}P(A_i)$
\end{enumerate}

\end{frame}


%----------------------------------------------------------------------------------------

\begin{frame}
\frametitle{Derived Properties}

Here are some things you can make sure are true using only the stuff from the previous slide:
\newline

\begin{enumerate}
\item $P(\varnothing) = 0$
\item $P(\bigcup_{i=1}^n A_i) = \sum_{i=1}^n P(A_i)$ if $A_1, \ldots, A_n$ are all disjoint
\item $P(A') = 1-P(A)$ for any $A$
\item if $A \subset B$ then $P(A) \le P(B)$
\item (in particular) for any $A$, $P(A) \le 1$
\item for any $A$ and $B$ (not necessarily disjoint), $P(A \cup B) = P(A) + P(B) - P(A \cap B)$
\end{enumerate}


\end{frame}

%----------------------------------------------------------------------------------------

\begin{frame}
\frametitle{Note}

Keep in mind that we are talking about a general $P(\cdot)$. In practice, you don't know it. However, every possibility must satisfy all these rules we have talked about.

\end{frame}

%----------------------------------------------------------------------------------------

%----------------------------------------------------------------------------------------



\end{document} 