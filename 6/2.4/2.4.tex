\documentclass{beamer}

\mode<presentation> {

%\usetheme{default}
%\usetheme{AnnArbor}
%\usetheme{Antibes}
%\usetheme{Bergen}
%\usetheme{Berkeley}
%\usetheme{Berlin}
%\usetheme{Boadilla}
%\usetheme{CambridgeUS}
%\usetheme{Copenhagen}
%\usetheme{Darmstadt}
%\usetheme{Dresden}
%\usetheme{Frankfurt}
%\usetheme{Goettingen}
%\usetheme{Hannover}
%\usetheme{Ilmenau}
%\usetheme{JuanLesPins}
%\usetheme{Luebeck}
\usetheme{Madrid}
%\usetheme{Malmoe}
%\usetheme{Marburg}
%\usetheme{Montpellier}
%\usetheme{PaloAlto}
%\usetheme{Pittsburgh}
%\usetheme{Rochester}
%\usetheme{Singapore}
%\usetheme{Szeged}
%\usetheme{Warsaw}


%\usecolortheme{albatross}
%\usecolortheme{beaver}
%\usecolortheme{beetle}
%\usecolortheme{crane}
%\usecolortheme{dolphin}
%\usecolortheme{dove}
%\usecolortheme{fly}
%\usecolortheme{lily}
%\usecolortheme{orchid}
%\usecolortheme{rose}
%\usecolortheme{seagull}
%\usecolortheme{seahorse}
%\usecolortheme{whale}
%\usecolortheme{wolverine}

%\setbeamertemplate{footline} % To remove the footer line in all slides uncomment this line
%\setbeamertemplate{footline}[page number] % To replace the footer line in all slides with a simple slide count uncomment this line

%\setbeamertemplate{navigation symbols}{} % To remove the navigation symbols from the bottom of all slides uncomment this line
}

\usepackage{graphicx} % Allows including images
\usepackage{booktabs} % Allows the use of \toprule, \midrule and \bottomrule in tables
\usepackage{amsfonts}
\usepackage{mathrsfs}
\usepackage{amsmath,amssymb,graphicx}

%----------------------------------------------------------------------------------------
%	TITLE PAGE
%----------------------------------------------------------------------------------------

\title["2.4"]{2.4: Conditional Probability}

\author{Taylor} 
\institute[UVA] 
{
University of Virginia \\
\medskip
\textit{} 
}
\date{} 

\begin{document}
%----------------------------------------------------------------------------------------

\begin{frame}
\titlepage 
\end{frame}
%----------------------------------------------------------------------------------------

\begin{frame}
\frametitle{Motivation}

Sometimes the probabilities of events change with the set of available information that we have. 
\newline

Example: Let $A = \{\text{Google's share price increases tomorrow}\}$. Then $P(A) \approx .5$
\newline

But if we let $C = \{\text{the government bans access to google.com}\}$, what's $P(A \text{ given } C)$?
\end{frame}

%----------------------------------------------------------------------------------------

\begin{frame}
\frametitle{Conditional Probability}

The notation: we'll write  $P(A \text{ given } C)$ like $P(A|C)$. And here's the formula:

\[
P(A|C) = \frac{P(A \cap C)}{P(C)}
\]

\end{frame}

%----------------------------------------------------------------------------------------


\begin{frame}
\frametitle{Note}

\begin{enumerate}
\item It helps to think of this in the context of venn diagrams
\item $C \neq \varnothing$, otherwise we're dividing by $0$ using this definition
\item sometimes we'll use  $P(A|C)P(C) = P(A \cap C)$

\end{enumerate}


\end{frame}

%----------------------------------------------------------------------------------------


\begin{frame}
\frametitle{Example 2.26 (pg. 76)}

\begin{align*}
P(A | B \cup C) &= \frac{P(A \cap (B \cup C))}{P(B \cup C)} \\
&= \frac{P[(A \cap B) \cup (A \cap C)]}{P(B \cup C)} \\
&= \frac{P(A \cap B) + P(A \cap C) - P((A \cap B) \cap (A \cap C)) }{P(B \cup C)} \\
&= \frac{P(A \cap B) + P(A \cap C) - P(A \cap B \cap A \cap C) }{P(B \cup C)} \\
&= \frac{P(A \cap B) + P(A \cap C) - P(A \cap B \cap C) }{P(B) + P(C) - P(B \cap C)} \\
&= .255
\end{align*}



\end{frame}

%----------------------------------------------------------------------------------------


\begin{frame}
\frametitle{Example}

Remember we said we would use $P(A|C)P(C) = P(A \cap C)$ a lot? This is useful in situations where there is some sort of sequential thing going on, or there are several stages of some random process.
\newline

Example: you're only interested in if the stock market goes up or down. The events that it goes up on Monday, Tuesday, Wednesday, Thursday and Friday are $A_1, A_2, A_3, A_4$ and $A_5$, respectively. The changes that it goes up or goes down only depend on what happened the day before (unless it's a Monday in which case let's assume it goes up or down with probability .5). Say there's a 75\% chance that it does the same thing as it did the day before. This means that there is a 25\% chance it does the opposite. What's the probability the market goes up every day of the week?

\end{frame}

%----------------------------------------------------------------------------------------


\begin{frame}
\frametitle{Example}

\begin{align*}
P(\text{goes up every day}) &= P(A_5 \cap A_4 \cap \cdots \cap A_1) \\
&= P(A_5 | A_4 \cap A_3 \cap A_2 \cap A_1)P(A_4 |A_3 \cap A_2 \cap A_1) \times \\
&      P(A_3 |  A_2 \cap A_1) P(A_2 | A_1) P(A_1) \\
&= P(A_5 | A_4)P(A_4 |A_3)P(A_3 |  A_2) P(A_2 | A_1) P(A_1) \\
&= (.75)(.75)(.75)(.75)(.5)
\end{align*}

\end{frame}

%----------------------------------------------------------------------------------------


\begin{frame}
\frametitle{Motivation}

Conditional probability is also a precursor for a thing called \textbf{Bayes' Theorem}. We need a few more ideas before we get there though...
\newline

If $A_1, \ldots, A_k$ are all pairwise disjoint and if $\bigcup_{i=1}^k A_i = \mathcal{S}$, then 
\[
P(B) = \sum_{i=1}^k P(B|A_i)P(A_i)
\]




\end{frame}

%----------------------------------------------------------------------------------------



\begin{frame}
\frametitle{Proof}

Notice that

\begin{align*}
P(B) &= P(B \cap \mathcal{S}) \\
&= P(B \cap (\cup_{i=1}^k A_i)) \\
&= P\left(\cup_{i=1}^k (A_i \cap B) \right) \\
&= \sum_{i=1}^k P\left( A_i \cap B \right) \\
&= \sum_{i=1}^k P\left( B | A_i \right) P(A_i)
\end{align*}

\end{frame}

%----------------------------------------------------------------------------------------


\begin{frame}
\frametitle{Bayes' Rule}

Here's Bayes' theorem:

\[
P(A_j | B) = \frac{P(B | A_j) P(A_j) }{\sum_{i=1}^kP(B | A_i) P(A_i)}
\]

\end{frame}

%----------------------------------------------------------------------------------------

\begin{frame}
\frametitle{Note}

A key take-away here is that there's a switch thing going on...the left hand side has probabilities conditioning on $B$, whereas the right hand side has probabilities conditioning on $A$s.

\end{frame}

%----------------------------------------------------------------------------------------

\begin{frame}
\frametitle{Example}

Example 2.30 pops up in STAT 2020/2120 a lot...
\newline

Let $D$ be the event that a person has a rare disease. Let $T$ be the event that you test positive for this disease. The bottom of page 80 gives us $P(D) = .001$, $P(T|D) = .99$ and $P(T|D') = .02$. What's the probability that a person has the disease if he/she is told that he/she has it?
\pause 
\newline

\[
P(D|T) = \frac{P(T|D)P(D)}{P(T|D)P(D) + P(T|D')P(D')} = .047
\]

\end{frame}

%----------------------------------------------------------------------------------------




\end{document} 