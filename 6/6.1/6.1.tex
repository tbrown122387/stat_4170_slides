\documentclass{beamer}

\mode<presentation> {

%\usetheme{default}
%\usetheme{AnnArbor}
%\usetheme{Antibes}
%\usetheme{Bergen}
%\usetheme{Berkeley}
%\usetheme{Berlin}
%\usetheme{Boadilla}
%\usetheme{CambridgeUS}
%\usetheme{Copenhagen}
%\usetheme{Darmstadt}
%\usetheme{Dresden}
%\usetheme{Frankfurt}
%\usetheme{Goettingen}
%\usetheme{Hannover}
%\usetheme{Ilmenau}
%\usetheme{JuanLesPins}
%\usetheme{Luebeck}
\usetheme{Madrid}
%\usetheme{Malmoe}
%\usetheme{Marburg}
%\usetheme{Montpellier}
%\usetheme{PaloAlto}
%\usetheme{Pittsburgh}
%\usetheme{Rochester}
%\usetheme{Singapore}
%\usetheme{Szeged}
%\usetheme{Warsaw}


%\usecolortheme{albatross}
%\usecolortheme{beaver}
%\usecolortheme{beetle}
%\usecolortheme{crane}
%\usecolortheme{dolphin}
%\usecolortheme{dove}
%\usecolortheme{fly}
%\usecolortheme{lily}
%\usecolortheme{orchid}
%\usecolortheme{rose}
%\usecolortheme{seagull}
%\usecolortheme{seahorse}
%\usecolortheme{whale}
%\usecolortheme{wolverine}

%\setbeamertemplate{footline} % To remove the footer line in all slides uncomment this line
%\setbeamertemplate{footline}[page number] % To replace the footer line in all slides with a simple slide count uncomment this line

%\setbeamertemplate{navigation symbols}{} % To remove the navigation symbols from the bottom of all slides uncomment this line
}

\usepackage{graphicx} % Allows including images
\usepackage{booktabs} % Allows the use of \toprule, \midrule and \bottomrule in tables
\usepackage{amsfonts}
\usepackage{mathrsfs}
\usepackage{amsmath,amssymb,graphicx, bm}

%----------------------------------------------------------------------------------------
%	TITLE PAGE
%----------------------------------------------------------------------------------------

\title["6.1"]{6.1: ARIMA models for nonstationary series} 

\author{Taylor} 
\institute[UVA] 
{
University of Virginia \\
\medskip
\textit{} 
}
\date{} 

\begin{document}
%----------------------------------------------------------------------------------------

\begin{frame}
\titlepage 
\end{frame}

%----------------------------------------------------------------------------------------

\begin{frame}
\frametitle{Motivation}

This chapter is all about putting the ``I" in ARIMA. 


\end{frame}

%----------------------------------------------------------------------------------------

\begin{frame}
\frametitle{Definition}

\begin{block}{definition}
If $d$ is a nonnegative integer, then $\{X_t\}$ is an {\bf ARIMA(p,d,q) process } if $Y_t = (1-B)^dX_t$ is a causal ARMA(p,q) process.
\end{block}

\begin{enumerate}
\item $X_t$ now has $d$ unit roots: it satisfies $\phi^*(B)X_t = \phi(B)(1-B)^d X_t = \theta(B)Z_t $
\item We've been doing this already when we difference our log-prices
\end{enumerate}
\end{frame}
%----------------------------------------------------------------------------------------

\begin{frame}
\frametitle{Explanation}

``I" stands for integrated (summed) because 
\[
X_t = X_0 + \sum_{j=1}^t Y_j
\]
where each $Y_j$ is from some stationary ARMA process. Clearly
\[
Y_t = (1-B)X_t 
\]

\end{frame}

%----------------------------------------------------------------------------------------

\begin{frame}
\frametitle{Mean versus Intercept}

Writing the intercept or mean terms can be done in two different ways. Make sure you know what your software is giving you.
\newline

The equation
\[
\phi(B)(1-B)^d(X_t - \mu t^d / d!) = \theta(B)Z_t
\]
is the same as 
\[
\phi(B)(1-B)^dX_t = c + \theta(B)Z_t.
\]

What's $EX_t$? What about $\mu$. What is $c$?

\end{frame}

%----------------------------------------------------------------------------------------

\begin{frame}
\frametitle{A Common Bug}

\begin{block}{Warning!}
By default, stats::arima() does not include a mean term when you fit ARIMA models. forecast::Arima() does. Make sure you are not forcing your expected returns to be $0$. Fore more information, see 6.1.R
\end{block}


\end{frame}





\end{document} 