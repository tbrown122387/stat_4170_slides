\documentclass{beamer}

\mode<presentation> {

%\usetheme{default}
%\usetheme{AnnArbor}
%\usetheme{Antibes}
%\usetheme{Bergen}
%\usetheme{Berkeley}
%\usetheme{Berlin}
%\usetheme{Boadilla}
%\usetheme{CambridgeUS}
%\usetheme{Copenhagen}
%\usetheme{Darmstadt}
%\usetheme{Dresden}
%\usetheme{Frankfurt}
%\usetheme{Goettingen}
%\usetheme{Hannover}
%\usetheme{Ilmenau}
%\usetheme{JuanLesPins}
%\usetheme{Luebeck}
\usetheme{Madrid}
%\usetheme{Malmoe}
%\usetheme{Marburg}
%\usetheme{Montpellier}
%\usetheme{PaloAlto}
%\usetheme{Pittsburgh}
%\usetheme{Rochester}
%\usetheme{Singapore}
%\usetheme{Szeged}
%\usetheme{Warsaw}


%\usecolortheme{albatross}
%\usecolortheme{beaver}
%\usecolortheme{beetle}
%\usecolortheme{crane}
%\usecolortheme{dolphin}
%\usecolortheme{dove}
%\usecolortheme{fly}
%\usecolortheme{lily}
%\usecolortheme{orchid}
%\usecolortheme{rose}
%\usecolortheme{seagull}
%\usecolortheme{seahorse}
%\usecolortheme{whale}
%\usecolortheme{wolverine}

%\setbeamertemplate{footline} % To remove the footer line in all slides uncomment this line
%\setbeamertemplate{footline}[page number] % To replace the footer line in all slides with a simple slide count uncomment this line

%\setbeamertemplate{navigation symbols}{} % To remove the navigation symbols from the bottom of all slides uncomment this line
}

\usepackage{graphicx} % Allows including images
\usepackage{booktabs} % Allows the use of \toprule, \midrule and \bottomrule in tables
\usepackage{amsfonts}
\usepackage{mathrsfs}
\usepackage{amsmath,amssymb,graphicx, bm}

%----------------------------------------------------------------------------------------
%	TITLE PAGE
%----------------------------------------------------------------------------------------

\title["6.5"]{6.5: Seasonal ARIMA Models} 

\author{Taylor} 
\institute[UVA] 
{
University of Virginia \\
\medskip
\textit{} 
}
\date{} 

\begin{document}
%----------------------------------------------------------------------------------------

\begin{frame}
\titlepage 
\end{frame}

%----------------------------------------------------------------------------------------

\begin{frame}
\frametitle{Motivation}

These are useful when you have a seasonal component of period $s$.
\newline

\end{frame}

%----------------------------------------------------------------------------------------

\begin{frame}
\frametitle{Definition}

\begin{block}{SARIMA(p,d,q)$\times$(P,D,Q)$_s$}
If $d$ and $D$ are nonnegative integers, then $\{X_t\}$ is a {\bf seasonal ARIMA(p,d,q)(P,D,Q) process with period $s$} if the differenced series $Y_t = (1-B)^d(1-B^s)^DX_t $ is a causal ARMA process defined by
\[
\phi(B)\Phi(B^s)Y_t = \theta(B) \Theta(B^s)Z_t, \hspace{10mm} \{Z_t\} \sim WN(0,\sigma^2),
\]
where $\phi(z) = 1 - \phi_1 z - \cdots - \phi_p z^p$, $\Phi(z) = 1 - \Phi_1 z - \cdots - \Phi_P z^P$, $\theta(z) = 1 + \theta_1 z + \cdots + \theta_q z^q$, $\Theta(z) = 1 + \Theta_1 z + \cdots \Theta_Q z^Q$.

\end{block}

\end{frame}

%----------------------------------------------------------------------------------------

\begin{frame}
\frametitle{Note}

\begin{block}{Quote From Book}
In Section 1.5 we discussed the classical decomposition model incorporating trend, seasonality, and random noise, namely, 
\[
X_t = m_t + s_t + Y_t .
\]
In modeling real data it might not be reasonable to assume, as in the classical decomposition model, that the seasonal component $s_t$ repeats itself precisely in the same way cycle after cycle. Seasonal ARIMA models allow for randomness in the seasonal pattern from one cycle to the next."
\end{block}

\end{frame}



%----------------------------------------------------------------------------------------

\begin{frame}
\frametitle{Example 1}
\[
\phi(B)\Phi(B^s)Y_t = \theta(B) \Theta(B^s)Z_t
\]
\[
X_t = U_t - .4 U_{t-12}
\]

Q: What are $p,d,q,P,D,Q,s$?
\newline
\pause

A: 
\begin{itemize}
\item $p=0$
\item $q=0$
\item $d=0$
\item $P=0$
\item $D=0$
\item $Q=1$
\item $s=12$
\end{itemize}

\end{frame}

%----------------------------------------------------------------------------------------

\begin{frame}
\frametitle{Example 1}
\[
\phi(B)\Phi(B^s)Y_t = \theta(B) \Theta(B^s)Z_t
\]
\[
X_t - .7 X_{t-12} = U_t
\]

Q: What are $p,d,q,P,D,Q,s$?
\newline
\pause

A: 
\begin{itemize}
\item $p=0$
\item $q=0$
\item $d=0$
\item $P=1$
\item $D=0$
\item $Q=0$
\item $s=12$
\end{itemize}

\end{frame}

%----------------------------------------------------------------------------------------

\begin{frame}
\frametitle{Example 1}
\[
\phi(B)\Phi(B^s)Y_t = \theta(B) \Theta(B^s)Z_t
\]
\[
X_t = U_t - .4 U_{t-12}
\]

Q: What is the ACF?
\newline
\pause

\[
\rho(\pm 12) = \frac{ \gamma(12) }{\gamma(0) } = \frac{-.4 \sigma^2 }{\sigma^2(1 + .4^2) } = -.4/(1+.4^2).
\]
And it equals zero everywhere else (besides at $1$).

\end{frame}
%----------------------------------------------------------------------------------------

\begin{frame}
\frametitle{Example 2}
\[
\phi(B)\Phi(B^s)Y_t = \theta(B) \Theta(B^s)Z_t
\]
\[
X_t - .7 X_{t-12} = U_t
\]

Q: What is the ACF?
\newline
\pause

For $k \ge 1$, 
\[
\gamma(12k) = \text{Cov}(X_{t+12k},X_t) = \text{Cov}(.7 X_{t+12(k-1)},X_t) = .7 \gamma(12(k-1))
\]
So 
\[
\rho(12k) = .7\rho(12[k-1])
\]
as well.

\end{frame}

%----------------------------------------------------------------------------------------

\begin{frame}
\frametitle{Example 3}

Consider the model
\[
(1-\phi B)(1- \Phi B^{12})X_t = Z_t.
\]
This can be rewritten as 
\[
X_t =  \phi X_{t-1}  +\Phi X_{t-12} - \phi \Phi X_{t-13} = Z_t.
\]
This is an AR(13) model where a lot of the coefficients are $0$, and you're only estimating three parameters.

\end{frame}


%----------------------------------------------------------------------------------------

\begin{frame}
\frametitle{Identification by Hand}

The book suggests:
\begin{enumerate}
\item first find $d,D,s$ so that $(1-B)^d(1-B^s)^DX_t$ is stationary
\item then look at ACF and PACF, and look out for significant bumps at lags of multiple $s$
\end{enumerate}

AICc works, too. But you should try to narrow it down to a few models by hand, first.
\newline

See code file for example.
\end{frame}

\end{document} 