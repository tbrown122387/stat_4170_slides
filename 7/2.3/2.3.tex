\documentclass{beamer}

\mode<presentation> {

%\usetheme{default}
%\usetheme{AnnArbor}
%\usetheme{Antibes}
%\usetheme{Bergen}
%\usetheme{Berkeley}
%\usetheme{Berlin}
%\usetheme{Boadilla}
%\usetheme{CambridgeUS}
%\usetheme{Copenhagen}
%\usetheme{Darmstadt}
%\usetheme{Dresden}
%\usetheme{Frankfurt}
%\usetheme{Goettingen}
%\usetheme{Hannover}
%\usetheme{Ilmenau}
%\usetheme{JuanLesPins}
%\usetheme{Luebeck}
\usetheme{Madrid}
%\usetheme{Malmoe}
%\usetheme{Marburg}
%\usetheme{Montpellier}
%\usetheme{PaloAlto}
%\usetheme{Pittsburgh}
%\usetheme{Rochester}
%\usetheme{Singapore}
%\usetheme{Szeged}
%\usetheme{Warsaw}


%\usecolortheme{albatross}
%\usecolortheme{beaver}
%\usecolortheme{beetle}
%\usecolortheme{crane}
%\usecolortheme{dolphin}
%\usecolortheme{dove}
%\usecolortheme{fly}
%\usecolortheme{lily}
%\usecolortheme{orchid}
%\usecolortheme{rose}
%\usecolortheme{seagull}
%\usecolortheme{seahorse}
%\usecolortheme{whale}
%\usecolortheme{wolverine}

%\setbeamertemplate{footline} % To remove the footer line in all slides uncomment this line
%\setbeamertemplate{footline}[page number] % To replace the footer line in all slides with a simple slide count uncomment this line

%\setbeamertemplate{navigation symbols}{} % To remove the navigation symbols from the bottom of all slides uncomment this line
}

\usepackage{graphicx} % Allows including images
\usepackage{booktabs} % Allows the use of \toprule, \midrule and \bottomrule in tables
\usepackage{amsfonts}
\usepackage{mathrsfs}
\usepackage{amsmath,amssymb,graphicx}

%----------------------------------------------------------------------------------------
%	TITLE PAGE
%----------------------------------------------------------------------------------------

\title["2.3"]{2.3: Counting Techniques}

\author{Taylor} 
\institute[UVA] 
{
University of Virginia \\
\medskip
\textit{} 
}
\date{} 

\begin{document}
%----------------------------------------------------------------------------------------

\begin{frame}
\titlepage 
\end{frame}
%----------------------------------------------------------------------------------------

\begin{frame}
\frametitle{Motivation}

If each outcome is equally likely, and there are $N$ distinct (think disjoint) outcomes, then the probability of any outcome $O$ is $P(O) = \frac{1}{N}$. Then computing probabilities of events is basically just counting up how many outcomes are in that event (i.e. $P(A) = \frac{\text{num. outcomes in A}}{N}$) (Note: same thing as page 66) (also: why can we add probabilities like this?)
\newline

When you have discrete random variables, counting rules help you find out how big the sample space is and/or how many outcomes are in your event in question.

\end{frame}

%----------------------------------------------------------------------------------------

\begin{frame}
\frametitle{Product Rule for Ordered Pairs}

If you have an ordered k-tuple of $k$ elements ($O_1, O_2, \ldots O_k$), where the $i$th element can be arranged $n_i$ ways where $i = 1, \ldots, k$, then the total number of possible tuples is 
\[
n_1 \times n_2 \times \cdots \times n_k
\]


\end{frame}

%----------------------------------------------------------------------------------------


\begin{frame}
\frametitle{Product Rule for Ordered Pairs}

Example: 
how many 4-digit passcodes are there for your phone?
\newline

how many 4-digit passcodes are there that start with $5$?
\newline

If you know that your friends passcode starts with a $5$, what is the chance that you can guess it correctly in one try?  You will have to assume that all the passcodes are equally likely.
\end{frame}

%----------------------------------------------------------------------------------------


\begin{frame}
\frametitle{Product Rule for Ordered Pairs}

Note(s):
\begin{enumerate}
\item The author recommends drawing tree diagrams to visualize situations like these.
\item The previous scenario's drawing mechanism is sometimes described as being \emph{with replacement} since the number of ways the $i$th element can occur doesn't affect subsequent or previous draws
\item now we'll talk about draws that are made \emph{without replacement}
\item we'll do a few examples to make this idea clearer
\end{enumerate}


\end{frame}

%----------------------------------------------------------------------------------------


\begin{frame}
\frametitle{Permutations}

What if instead you were taking $k$ things from $n$, and when you took an item, it couldn't be chosen again (e.g. people picking a seat at a table). 
\newline

Any ordered sequence of $k$ objects taken from a set of $n$ distinct objects is called a \textbf{permutation}

\end{frame}

%----------------------------------------------------------------------------------------


\begin{frame}
\frametitle{Permutations}

Example 2.2.1 from page 69: 10 teaching assistants are available. The professor needs a TA to grade exactly one problem each on a 4 problem test. How many ways can he pick TAs to grade his problems?

\[
\text{number of ways} = 10 * (10 - 1) * (10 - 2) * (10 - 3)
\]
\end{frame}

%----------------------------------------------------------------------------------------


\begin{frame}
\frametitle{Permutations}

In general we call the number of permutatons of $k$ things from $n$ $P_{k,n}$. It's formula is:

\[
P_{k,n} = n(n-1)\cdots(n-k+2)(n-k+1) = \frac{n!}{(n-k)!}
\]

If you haven't seen factorials before: $m! = (m)(m-1)\cdots(2)(1)$
\end{frame}

%----------------------------------------------------------------------------------------



\begin{frame}
\frametitle{Combinations}

We just highlighted the distinction between \emph{with replacement} and \emph{without replacement}.
\newline

Now another distinction: \textbf{ordered} and \textbf{unordered}
\newline

Example: in the previous problem the professor cared which TA graded which problem; in other words, the order mattered. What if he only cared about which TAs he chose, and he didn't care about what assignment they had?

\end{frame}

%----------------------------------------------------------------------------------------


\begin{frame}
\frametitle{Combinations}

Given a set of $n$ objects, any unordered subset of size $k$ that can be formed is called a \textbf{combination}
\newline

The number of combinations of size $k$ from $n$ things is often denoted by the ${n \choose k}$, read as ``$n$ choose $k$" and its formula is 

\[
{n \choose k} = \frac{n!}{k!(n-k)!}
\]

\end{frame}

%----------------------------------------------------------------------------------------

\begin{frame}
\frametitle{a connection...}

When there is \emph{no replacement}, the connection between the number of ordered and unordered things is 
\[
P_{k,n} = k! {n \choose k}
\]

\end{frame}

%----------------------------------------------------------------------------------------

\begin{frame}
\frametitle{Combinations}

Example: how many 5-card poker hands are there in a 52 card deck?

\end{frame}

%----------------------------------------------------------------------------------------




\end{document} 