\documentclass{beamer}

\mode<presentation> {

%\usetheme{default}
%\usetheme{AnnArbor}
%\usetheme{Antibes}
%\usetheme{Bergen}
%\usetheme{Berkeley}
%\usetheme{Berlin}
%\usetheme{Boadilla}
%\usetheme{CambridgeUS}
%\usetheme{Copenhagen}
%\usetheme{Darmstadt}
%\usetheme{Dresden}
%\usetheme{Frankfurt}
%\usetheme{Goettingen}
%\usetheme{Hannover}
%\usetheme{Ilmenau}
%\usetheme{JuanLesPins}
%\usetheme{Luebeck}
\usetheme{Madrid}
%\usetheme{Malmoe}
%\usetheme{Marburg}
%\usetheme{Montpellier}
%\usetheme{PaloAlto}
%\usetheme{Pittsburgh}
%\usetheme{Rochester}
%\usetheme{Singapore}
%\usetheme{Szeged}
%\usetheme{Warsaw}


%\usecolortheme{albatross}
%\usecolortheme{beaver}
%\usecolortheme{beetle}
%\usecolortheme{crane}
%\usecolortheme{dolphin}
%\usecolortheme{dove}
%\usecolortheme{fly}
%\usecolortheme{lily}
%\usecolortheme{orchid}
%\usecolortheme{rose}
%\usecolortheme{seagull}
%\usecolortheme{seahorse}
%\usecolortheme{whale}
%\usecolortheme{wolverine}

%\setbeamertemplate{footline} % To remove the footer line in all slides uncomment this line
%\setbeamertemplate{footline}[page number] % To replace the footer line in all slides with a simple slide count uncomment this line

%\setbeamertemplate{navigation symbols}{} % To remove the navigation symbols from the bottom of all slides uncomment this line
}

\usepackage{graphicx} % Allows including images
\usepackage{booktabs} % Allows the use of \toprule, \midrule and \bottomrule in tables
\usepackage{amsfonts}
\usepackage{mathrsfs}
\usepackage{amsmath,amssymb,graphicx}

%----------------------------------------------------------------------------------------
%	TITLE PAGE
%----------------------------------------------------------------------------------------

\title["7.1"]{7.1: Historical Overview} 

\author{Taylor} 
\institute[UVA] 
{
University of Virginia \\
\medskip
\textit{} 
}
\date{} 

\begin{document}
%----------------------------------------------------------------------------------------

\begin{frame}
\titlepage 
\end{frame}
%----------------------------------------------------------------------------------------

\begin{frame}
\frametitle{Definitions}

\begin{itemize}
\item $P_t$: the closing price on day/week/month $t$ of stock/stock index
\item $X_t = \log P_t$: log-price...observed paths look very much like those of a random walk 
\item $Z_t = \triangledown X_t$: the log return
\item $100 Z_t$: percentage returns
\item $h_t = \text{Var}(Z_t|Z_{1:t-1})$ : the conditional variance aka volatility
\end{itemize}


\end{frame}

%----------------------------------------------------------------------------------------

\begin{frame}
\frametitle{Motivation}

$Z_t$ sometimes looks roughly stationary. The ACF rarely gives any clear or consistent ARMA model recommendations. Also, if we fit some ARMA model, $h_t$ is independent of $t$ and independent of $Z_{1:t-1}$\footnote{homework question}. This assumption is often violated.
\newline

We would like models that take into account stylized features that appear such as {\bf tail-heaviness, asymmetry, volatility clustering and serial dependence without correlation}. We introduce AutoRegressive Conditional Heteroscedasticity (ARCH) models, Generalized AutoRegressive Conditional Heteroscedasticity (GARCH), and stochastic volatility (SVOL) models in the next slide.

\end{frame}

%----------------------------------------------------------------------------------------

\begin{frame}
\frametitle{Definitions}

\begin{block}{ARCH(p)}
\begin{align*}
Z_t &= \sqrt{h_t}e_t, \hspace{10mm} \{e_t\} \sim IID(0,1) \\
h_t &= \alpha_0 + \sum_{i=1}^p \alpha_i Z^2_{t-i}.
\end{align*}
$\alpha_0 > 0$, $a_i \ge 0$, $p \in \mathbb{N}$
\end{block}

\begin{itemize}
\item volatility increases if we have observed big movements
\item setting $a_i$ to $0$ gives us white noise model
\item $\{e_t\}$ is sometimes but not always assumed to be Normal
\end{itemize}

\end{frame}



%----------------------------------------------------------------------------------------

\begin{frame}
\frametitle{Definitions}

\begin{block}{GARCH(p,q)}
\begin{align*}
Z_t &= \sqrt{h_t}e_t, \hspace{10mm} \{e_t\} \sim IID(0,1) \\
h_t &= \alpha_0 + \sum_{i=1}^p \alpha_i Z^2_{t-i} + \sum_{j=1}^q \beta_j h_{t-j}.
\end{align*}
$\alpha_0 > 0$, $a_i \ge 0$, $\beta_j \ge 0$, $p \in \mathbb{N}$
\end{block}

\begin{itemize}
\item now volatility is a function of its own past values, in addition to the past observations
\item $\{e_t\}$ may or may not be normal
\end{itemize}


\end{frame}

%----------------------------------------------------------------------------------------

\begin{frame}
\frametitle{Definitions}

\begin{block}{SVOL}
\begin{align*}
Z_t &= \sqrt{h_t}e_t, \hspace{10mm} \{e_t\} \sim IID(0,1) \\
\ln h_t &= \gamma_0 + \gamma_1 \ln h_{t-1} + \eta_t, \hspace{10mm} \{\eta_t\} \sim IID(0,\sigma^2). 
\end{align*}
where $\{\eta_t\}$ and $\{e_t\}$ are independent.
\end{block}

\begin{itemize}
\item log-volatility is an AR(1) process. 
\item $\gamma_1$ is usually around $.95$.
\item more difficult to estimate (can't even evaluate likelihood)
\end{itemize}


\end{frame}

\end{document} 