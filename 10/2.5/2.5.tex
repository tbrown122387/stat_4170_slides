\documentclass{beamer}

\mode<presentation> {

%\usetheme{default}
%\usetheme{AnnArbor}
%\usetheme{Antibes}
%\usetheme{Bergen}
%\usetheme{Berkeley}
%\usetheme{Berlin}
%\usetheme{Boadilla}
%\usetheme{CambridgeUS}
%\usetheme{Copenhagen}
%\usetheme{Darmstadt}
%\usetheme{Dresden}
%\usetheme{Frankfurt}
%\usetheme{Goettingen}
%\usetheme{Hannover}
%\usetheme{Ilmenau}
%\usetheme{JuanLesPins}
%\usetheme{Luebeck}
\usetheme{Madrid}
%\usetheme{Malmoe}
%\usetheme{Marburg}
%\usetheme{Montpellier}
%\usetheme{PaloAlto}
%\usetheme{Pittsburgh}
%\usetheme{Rochester}
%\usetheme{Singapore}
%\usetheme{Szeged}
%\usetheme{Warsaw}


%\usecolortheme{albatross}
%\usecolortheme{beaver}
%\usecolortheme{beetle}
%\usecolortheme{crane}
%\usecolortheme{dolphin}
%\usecolortheme{dove}
%\usecolortheme{fly}
%\usecolortheme{lily}
%\usecolortheme{orchid}
%\usecolortheme{rose}
%\usecolortheme{seagull}
%\usecolortheme{seahorse}
%\usecolortheme{whale}
%\usecolortheme{wolverine}

%\setbeamertemplate{footline} % To remove the footer line in all slides uncomment this line
%\setbeamertemplate{footline}[page number] % To replace the footer line in all slides with a simple slide count uncomment this line

%\setbeamertemplate{navigation symbols}{} % To remove the navigation symbols from the bottom of all slides uncomment this line
}

\usepackage{graphicx} % Allows including images
\usepackage{booktabs} % Allows the use of \toprule, \midrule and \bottomrule in tables
\usepackage{amsfonts}
\usepackage{mathrsfs}
\usepackage{amsmath,amssymb,graphicx}

%----------------------------------------------------------------------------------------
%	TITLE PAGE
%----------------------------------------------------------------------------------------

\title["2.5"]{2.5: Independence}

\author{Taylor} 
\institute[UVA] 
{
University of Virginia \\
\medskip
\textit{} 
}
\date{} 

\begin{document}
%----------------------------------------------------------------------------------------

\begin{frame}
\titlepage 
\end{frame}
%----------------------------------------------------------------------------------------

\begin{frame}
\frametitle{Motivation}

When we were talking about conditional probability before we usually talked about examples where either $P(A) > P(A|B)$ or $P(A) < P(A|B)$. Quite often we'll assume that this isn't true when we're putting together statistical models, though. 

\end{frame}

%----------------------------------------------------------------------------------------

\begin{frame}
\frametitle{Definition}

Two events $A$ and $B$ are \textbf{independent} if $P(A|B) = P(A)$.
\newline

Two events $A$ and $B$ are \textbf{dependent} if $P(A|B) \neq P(A)$

\end{frame}

%----------------------------------------------------------------------------------------


\begin{frame}
\frametitle{Motivation}

The following are equivalent (you should check all of these):
\begin{enumerate}
\item $P(A|B) = P(A)$
\item $P(A|B') = P(A)$
\item $P(A'|B) = P(A')$
\item $P(A'|B') = P(A')$
\end{enumerate}
\pause

Also, these are too
\begin{enumerate}
\item $P(A|B) = P(A)$
\item $P(B|A) = P(B)$
\end{enumerate}

\end{frame}

%----------------------------------------------------------------------------------------


\begin{frame}
\frametitle{Another definition}

Another definition:
\newline

Events $A$ and $B$ are \textbf{independent} if 
\[
P(A \cap B) = P(A)P(B)
\]

this one is less intuitive but more general; it holds when we're talking about events that might have probability $0$. It also extends more easily to when we talk about independence between more than two things at a time. 


\end{frame}

%----------------------------------------------------------------------------------------


\begin{frame}
\frametitle{Definition}

$A_1, \ldots A_n$ are \textbf{mutually independent} if for every $k = 1, 2, \ldots, n$ and every subset of indices you can make with such a $k$ $i_1, \ldots, i_k$
\[
P(A_{i_1} \cap \cdots \cap A_{i_k}) = P(A_{i_1}) \times \cdots \times P(A_{i_k})
\]
\pause

Note that this is stronger than pairwise independence...


\end{frame}

%----------------------------------------------------------------------------------------


\begin{frame}
\frametitle{Example}

Let's try number 70 on page 89.
\pause
\newline

Let $T_1$ and $T_2$ be the blood types for the first and second persons. Then $P(\{T_1=O\} \cap \{T_2=0\}) = P(\{T_1=O\}) P(\{T_2=O\})= .44^2 = .1936$
\pause
\newline


\begin{align*}
P(\text{both match}) &= P(\text{both O or both A or both B or both AB})\\
&= P\left[ (\{T_1 = O\} \cap \{T_2 = O\}) \cup  \cdots \right] \\
&= P(\{T_1 = O\} \cap \{T_2 = O\}) + \cdots  \\
&= P(\{T_1 = O\}) P(\{T_2 = O\}) + \cdots \\
&= .42^2 + .1^2 + .04^2 + .44^2 \\
&= .3816
\end{align*}

\end{frame}


%----------------------------------------------------------------------------------------



\end{document} 