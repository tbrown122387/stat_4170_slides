\documentclass{beamer}

\mode<presentation> {

%\usetheme{default}
%\usetheme{AnnArbor}
%\usetheme{Antibes}
%\usetheme{Bergen}
%\usetheme{Berkeley}
%\usetheme{Berlin}
%\usetheme{Boadilla}
%\usetheme{CambridgeUS}
%\usetheme{Copenhagen}
%\usetheme{Darmstadt}
%\usetheme{Dresden}
%\usetheme{Frankfurt}
%\usetheme{Goettingen}
%\usetheme{Hannover}
%\usetheme{Ilmenau}
%\usetheme{JuanLesPins}
%\usetheme{Luebeck}
\usetheme{Madrid}
%\usetheme{Malmoe}
%\usetheme{Marburg}
%\usetheme{Montpellier}
%\usetheme{PaloAlto}
%\usetheme{Pittsburgh}
%\usetheme{Rochester}
%\usetheme{Singapore}
%\usetheme{Szeged}
%\usetheme{Warsaw}


%\usecolortheme{albatross}
%\usecolortheme{beaver}
%\usecolortheme{beetle}
%\usecolortheme{crane}
%\usecolortheme{dolphin}
%\usecolortheme{dove}
%\usecolortheme{fly}
%\usecolortheme{lily}
%\usecolortheme{orchid}
%\usecolortheme{rose}
%\usecolortheme{seagull}
%\usecolortheme{seahorse}
%\usecolortheme{whale}
%\usecolortheme{wolverine}

%\setbeamertemplate{footline} % To remove the footer line in all slides uncomment this line
%\setbeamertemplate{footline}[page number] % To replace the footer line in all slides with a simple slide count uncomment this line

%\setbeamertemplate{navigation symbols}{} % To remove the navigation symbols from the bottom of all slides uncomment this line
}

\usepackage{graphicx} % Allows including images
\usepackage{booktabs} % Allows the use of \toprule, \midrule and \bottomrule in tables
\usepackage{amsfonts}
\usepackage{mathrsfs}
\usepackage{amsmath,amssymb,graphicx}

%----------------------------------------------------------------------------------------
%	TITLE PAGE
%----------------------------------------------------------------------------------------

\title["3.1"]{3.1: ARMA(p,q) models} 

\author{Taylor} 
\institute[UVA] 
{
University of Virginia \\
\medskip
\textit{} 
}
\date{} 

\begin{document}
%----------------------------------------------------------------------------------------

\begin{frame}
\titlepage 
\end{frame}
%----------------------------------------------------------------------------------------

\begin{frame}
\frametitle{Motivation}

We extend ARMA models in this chapter.

\end{frame}

%----------------------------------------------------------------------------------------

\begin{frame}
\frametitle{Definitions}

\begin{block}{ARMA(p,q) process}
$\{X_t\}$ is an ARMA(p,q) process if $\{X_t\}$ is stationary and if for every $t$,
\[
X_t - \phi_1 X_{t-1} - \cdots - \phi_p X_{t-p} = Z_t + \theta_1 Z_{t-1} + \cdots + \theta_q Z_{t-q}
\]
where $\{Z_t\} \sim \text{WN}(0,\sigma^2)$ and the polynomials $\phi(z) = (1 - \phi_1 z - \cdots - \phi_p z^p)$ and $\theta(z) = (1 + \theta_1 z  + \cdots + \theta_qz^q)$ have no common factors.
\end{block}

\begin{enumerate}
\item $X_t$ is an ARMA(p,q) with mean $\mu$ if we replace the above $X_t$ with $X_t - \mu$
\item Easier to write $\phi(B)X_t = \theta(B)Z_t$ 
\item AR(p) if $\theta(z) = 1$
\item MA(q) if $\phi(z) = 1$
\end{enumerate}

\end{frame}


%----------------------------------------------------------------------------------------

\begin{frame}
\frametitle{From Old to New}

If we learn a little about complex numbers, we can start stating these parameter restrictions in terms of the argument to these polynomials, $z = a+bi$. This helps us because we're starting to deal with more parameters now, and because it's more common in practice. Recall $|z|^2 = z \bar{z} = a^2 + b^2$.
\newline

Before, in 2.3: $(1-\phi_1 z)$ was our AR(1) polynomial. Setting it equal to zero yields
\[
1 = \phi_1 z_1 \iff \phi_1 = 1/z_1 \iff z_1 = 1/\phi_1
\]
where $z_1$ is the root (there's only one and it's real in this case). 
\newline

The backshift operator becomes a complex number. Stationary solutions exist if and only if $\phi_1 \neq \pm 1$. Stated in terms of $z$ now $(1 - \phi_1 z) \neq 0$ when $|z| = \pm 1$. Why the $| \cdot |$? When we have higher order AR models, the roots are often complex.
\newline

\end{frame}

%----------------------------------------------------------------------------------------

\begin{frame}
\frametitle{Complex Polynomials}

Every degree $p$ polynomial $\phi(z)$ can be factored into 
\begin{align*}
\phi(z) &= 1 + \phi_1 z + \cdots \phi_p z^p \\
&= \phi_p(z - z_1)(z - z_2)\cdots(z-z_p).
\end{align*}
Because the coefficients $\phi_1, \ldots, \phi_p$ are real numbers, all of these roots $z_1,\ldots, z_p$, even though they are written in terms of our coefficients, are real or complex pairs.
\newline

Example:
\begin{align*}
\phi(z) &= 2 z^3 + 0z^2 + z + 0 \\
&= 2[z^3 + 0z^2 + \frac{1}{2}z + 0]  \\
&= 2(z-0)(z-\sqrt{-1/2})(z+\sqrt{-1/2}) \\
\end{align*}
Number of roots matches the degree of polynomial. One complex pair of roots, and one real root. $z_1 = 0$, $z_2 = \sqrt{-1/2}$ and $z_3 = -\sqrt{-1/2}$.

\end{frame}
%----------------------------------------------------------------------------------------

\begin{frame}
\frametitle{Unit Roots}

Stationarity from last chapter: $(1 - \phi z) \neq 0$ when $|z| = \pm 1$
\newline

For higher order AR models, the analagous situation is that $\phi(z)$ has no ``unit roots." That is, a stationary ARMA solution exists if and only if $\phi(z) = (1 - \phi_1 z - \cdots \phi_p z^p) \neq 0$ for $|z| = \pm 1$. 
\newline

Whenever you plug in a $z$ that lies on the unit circle in the complex plane, the polynomial will not be equal to $0$. Our definition of ARMA requires this because we want to divide both sides by $\phi(z)$. No unit roots implies that 
\[
\frac{1}{\phi(z)} = \sum_{j=-\infty}^{\infty}\chi_j z^j
\]
and $\sum_{j=-\infty}^{\infty}|\chi_j| < \infty$ for $1 - \delta < |z| < 1 + \delta$ with some $\delta > 0$ 

\end{frame}

%----------------------------------------------------------------------------------------

\begin{frame}
\frametitle{Causality $\to$ Stationarity }

\[
\frac{1}{\phi(z)} = \sum_{j=-\infty}^{\infty}\chi_j z^j
\]
and $\sum_{j=-\infty}^{\infty}|\chi_j| < \infty$ for $1 - \delta < |z| < 1 + \delta$ with some $\delta > 0$
\newline

This is a fancy way of saying we can divide both sides of the equation by $\phi(B)$ to isolate $X_t$ on the left hand side.
\newline

Notice this doesn't guarantee that it's causal. For that we need to further restrict the roots to lie outside the unit circle. 

\end{frame}

%----------------------------------------------------------------------------------------

\begin{frame}
\frametitle{Lemmas}
\begin{block}{stationarity}
A stationary solution of $\{X_t\}$ exists and is unique if and only if $|z| = 1$ implies
\[
\phi(z) = 1 - \phi_1 z - \phi_2 z^2 - \cdots - \phi_pz^p \neq 0.
\]
\end{block}

\begin{block}{causality}
An ARMA(p,q) process $\{X_t\}$ is {\bf causal} if and only if $|z| \le 1$ implies
\[
\phi(z) = 1 - \phi_1 z - \phi_2 z^2 - \cdots - \phi_pz^p \neq 0.
\]
\end{block}
\begin{block}{causality 2}
An ARMA(p,q) process $\{X_t\}$ is {\bf causal} if and only if there exist $\{\psi_j\}$ such that $\sum_{j=-\infty}^{\infty}|\psi_j| < \infty$ and $X_t = \sum_{j=0}^{\infty} \psi_j Z_{t-j} \text{ for all $t$}$
\end{block}


\end{frame}

%----------------------------------------------------------------------------------------

\begin{frame}[fragile]
\frametitle{Example 3.1.1}
Consider the ARMA(1,1) model 
\[
X_t - .5 X_{t-1} = Z_t + .4 Z_{t-1}
\]
Is it stationary? Is it causal?
\pause
\newline

We don't care about the MA part for this.
\[
1 - .5 z_1 = 0 \iff z_1 = 1/.5 \iff z_1 = 2
\]
$|2| \neq 0$ so a stationary solution exists. $|2| > 0$ so it is also causal. 

\begin{verbatim}
polyroot(c(1, -.5))
\end{verbatim}

\end{frame}

%----------------------------------------------------------------------------------------

\begin{frame}
\frametitle{Invertibility}

Invertibility involves the same math with complex polynomials, but only concerns $\theta(z)$.
\newline

\begin{block}{invertibility}
An ARMA(p,q) process is $\{X_t\}$ is {\bf invertible} if and only if there exist constants $\{\pi_j\}$ such that $\sum_{j=0}^{\infty}|\pi_j| < \infty$ and 
\[
Z_t = \sum_{j=0}^{\infty}\pi_j X_{t-j}.
\]
\end{block}
\begin{block}{invertibility 2}
An ARMA(p,q) process is $\{X_t\}$ is {\bf invertible} if and only if $|z| \le 1$ implies
\[
\theta(z) = 1 + \theta_1 z + \cdots + \theta_q z^q \neq 0.
\]
\end{block}

\end{frame}
%----------------------------------------------------------------------------------------

\begin{frame}[fragile]
\frametitle{Back to Example 3.1.1}
Consider the ARMA(1,1) model 
\[
X_t - .5 X_{t-1} = Z_t + .4 Z_{t-1}
\]
Is it invertible?
\pause
\newline

We don't care about the AR part for this.
\[
1 + .4 z_1 = 0 \iff z_1 = -1/.4 \iff z_1 = -2.5
\]
$|-2.5| > 0$ so it is invertible! 

\begin{verbatim}
polyroot(c(1,.4))
\end{verbatim}


\end{frame}

%----------------------------------------------------------------------------------------

\begin{frame}
\frametitle{Back to Example 3.1.1}
Consider the ARMA(1,1) model 
\[
X_t - .5 X_{t-1} = Z_t + .4 Z_{t-1}
\]
find $\psi(z)$ and $\pi(z)$.
\pause
\newline

$\chi(z)$ is the reciprocal of $\phi(z)$ iff
\begin{align*}
(1 + \chi_1 z + \chi_2 z^2 + \cdots)(1 - .5z) &= 1\\
1 + z(-.5 + \chi_1) + z^2(- \chi_1 .5 + \chi_2) + \cdots &= 1 + 0z + 0 z^2 + \cdots
\end{align*}
So $\chi_1 = 2^{-1}$, $\chi_2 = 2^{-2}, \ldots$
\end{frame}

%----------------------------------------------------------------------------------------

\begin{frame}
\frametitle{Back to Example 3.1.1}
Consider the ARMA(1,1) model 
\[
X_t - .5 X_{t-1} = Z_t + .4 Z_{t-1}
\]

Now that we have the reciprocal, we can multiply both sides by it
\begin{align*}
\psi(z) = \chi(z)\theta(z) &= (1 + z/2 + z^2/4 + \cdots)(1 + .4z) \\
&= 1 + z(.4 + .5) + z^2(.4\cdot.5+.5^2) + \cdots
\end{align*}

So 
\[
X_t = \sum_{j=0}^{\infty}\psi_j Z_{t-j} 
\]
where $\psi_0 = 1$, $\psi_1 = .4 + .5$, $\psi_2 = .5(.4 + .5)$, $\psi_3 = .5^2(.4 + .5)$, $\ldots$
\newline

\end{frame}

%----------------------------------------------------------------------------------------

\begin{frame}[fragile]
\frametitle{Back to Example 3.1.1}

Checking our answers:
\[
X_t - .5 X_{t-1} = Z_t + .4 Z_{t-1}
\]
\[
X_t = \sum_{j=0}^{\infty}\psi_j Z_{t-j} 
\]
where $\psi_0 = 1$, $\psi_1 = .4 + .5$, $\psi_2 = .5(.4 + .5)$, $\psi_3 = .5^2(.4 + .5)$, $\ldots$
\newline

This function ignores the $1$s. Also be careful about the signs of the coefficients!
\begin{verbatim}
ARMAtoMA(ar=c(.5), ma=c(.4), lag.max=3)
\end{verbatim}

\end{frame}



%----------------------------------------------------------------------------------------

\begin{frame}
\frametitle{Back to Example 3.1.1}
Consider the ARMA(1,1) model 
\[
X_t - .5 X_{t-1} = Z_t + .4 Z_{t-1}
\]

$\xi(z)$ is the reciprocal of $\theta(z)$ iff
\begin{align*}
\xi(z)\theta(z) &= 1  \\
(1 + \xi_1 z + \xi_2 z^2 + \cdots)(1 + .4 z) &= 1  + 0z + 0z^2 + \cdots\\
1 + z(.4 + \xi_1) + z^2 ( \xi_1 .4 + \xi_2) + z^3(.4 \xi_2 + \xi_3) + \cdots &= 1+ 0z + 0z^2 + \cdots
\end{align*}

So $\xi_0 = 1$, $\xi_1 = - .4$, $\xi_2 = .4^2$, $\xi_3 = -.4^3, \ldots$
\end{frame}
%----------------------------------------------------------------------------------------

\begin{frame}
\frametitle{Back to Example 3.1.1}
Consider the ARMA(1,1) model 
\[
X_t - .5 X_{t-1} = Z_t + .4 Z_{t-1}
\]

Now that we have the reciprocal, we can multiply both sides by it
\begin{align*}
\pi(z) = \xi(z)\phi(z) &= (1 + -.4z + .4^2 z + \cdots)(1 - .5z) \\
&= 1 + z(-.4 - .5) + z^2(.5\cdot.4+.4^2) + \cdots
\end{align*}

So 
\[
Z_t = \sum_{j=0}^{\infty}\pi_j X_{t-j} 
\]
where $\pi_0 = 1$, $\pi_1 = -(.4 + .5)$, $\pi_2 = -(.4 + .5)(-.4)$, $\pi_3 = -(.4 + .5)(-.4)^2$, $\ldots$
\end{frame}

%----------------------------------------------------------------------------------------

\begin{frame}
\frametitle{3.1.2}
Consider the AR(2) model 
\[
X_t = .7 X_{t-1} - .1X_{t-2} = Z_t 
\]
Is it causal? Find it's MA($\infty$) representation.
\pause

\begin{align*}
\phi(z) &= 1 - .7z + .1 z^2 \\
&= .1(z^2 - 7z + 10) \\
&= .1(z - 5)(z-2)
\end{align*}

Because both $|z_1| = |5|$ and $|z_2| = |2|$ are outside of the unit circle, this model is causal. 

\end{frame}

%----------------------------------------------------------------------------------------

\begin{frame}
\frametitle{3.1.2}
Consider the AR(2) model 
\[
X_t = .7 X_{t-1} - .1X_{t-2} = Z_t 
\]
Is it causal? Find it's MA($\infty$) representation.
\pause
\newline

Find the reciprocal of $\phi(z)$ first
\begin{align*}
\chi(z)\phi(z) &= 1\\
(1 + \chi_1 z + \chi_2 z^2 + \chi_3 z^3 + \cdots)(1 - .7z + .1 z^2 ) &= 1 \\
1 + z(-.7 + \chi_1) + z^2(.1 - .7\chi_1 + \chi_2) + z^3(.1 \chi_1 + -.7 \chi_2 + \chi_3) + \cdots &= 1
\end{align*}

So $\chi_0 = 1$, $\chi_1 = .7$, $\chi_2 = .7^2 -.1 $, $\chi_3 = .7 \chi_2  - .1 \chi_1$.
\end{frame}

%----------------------------------------------------------------------------------------

\begin{frame}
\frametitle{3.1.2}
Consider the AR(2) model 
\[
X_t = .7 X_{t-1} - .1X_{t-2} = Z_t 
\]
Is it causal? Find it's MA($\infty$) representation.
\newline

Then multiply both sides by this reciprocal 
\begin{align*}
\psi(z) &= \chi(z)\theta(z)\\
&= \chi(z)
\end{align*}


\end{frame}

%----------------------------------------------------------------------------------------

\begin{frame}
\frametitle{3.1.3}
Consider the ARMA(2,1) model 
\[
X_t - .75 X_{t-1} + .5625 X_{t-2} = Z_t + 1.25 Z_{t-1} 
\]
Is it causal? Is it invertible? Hint: $.75^2 = .5625$
\pause
\newline

Causality check on $\phi(z) = 1 - .75 z + .5625 z^2$ using quadratic formula:
\begin{align*}
\frac{.75 \pm \sqrt{.75^2 - 4(.5625) }}{(2)(.5625)} &= \frac{.75 \pm \sqrt{.75^2\{1 - 4\} }}{(2)(.5625)} \\
&= \frac{.75 \pm .75 \sqrt{-3 }}{(2)(.5625)} = \frac{2}{3}(1 \pm \sqrt{3}i)
\end{align*}
$|z_1| = |2/3 + i 2\sqrt{3}/3| = \sqrt{4/9 + 12/9} = 4/3 > 0$. $|z_2|$ is the same. So this is a causal model.

\end{frame}

%----------------------------------------------------------------------------------------

\begin{frame}
\frametitle{3.1.3}
Consider the ARMA(2,1) model 
\[
X_t - .75 X_{t-1} + .5625 X_{t-2} = Z_t + 1.25 Z_{t-1} 
\]
Is it causal? Is it invertible? Hint: $.75^2 = .5625$
\pause
\newline

Invertibility check on $\theta(z) = 1 + 1.25 z$. Clearly $|z_1| = |-4/5| <= 1$. So this is not invertible.

\end{frame}
\end{document} 